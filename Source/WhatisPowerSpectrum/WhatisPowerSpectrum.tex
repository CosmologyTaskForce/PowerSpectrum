% !TEX program = xelatex

%This document had would had been used on Ways to Singularity, which is a website that supports MathJax. So some html elements may occur in this document. DELETE THIS when publishing. (If you are not very clear on the grammar I used here, read the academic publication called Time Traveller's Handbook of 1001 Tense Formations by Dr Dan Streetmentioner, which had would had been publish in year 220010 of Gregorian Calendar.)

\documentclass[12pt,a4paper]{book}

\usepackage{amsthm,amsfonts,amssymb,bm}
\usepackage{mathrsfs}
\usepackage[fleqn]{amsmath}

% Page settings
\addtolength{\textheight}{2.0cm}
\addtolength{\voffset}{-2cm}
\addtolength{\hoffset}{-1.0cm}
\addtolength{\textwidth}{2.0cm}

%\allowdisplaybreaks

\usepackage{subeqnarray}
%\usepackage{color}
\usepackage[usenames,dvipsnames]{color}
\usepackage{url}
\usepackage{ulem}
\usepackage{indentfirst}   % Indent first line of a paragraph
%\usepackage{textcomp}

\usepackage{enumerate}

\usepackage{graphicx}
\usepackage[hang,small,bf]{caption}
\setlength{\captionmargin}{50pt}


%%Here is the configuration for chinese. setmainfont is the default font of the text.
\usepackage[cm-default]{fontspec}
\usepackage{xunicode}
%\usepackage{xltxtra}
\setmainfont{Arial}
%\setsansfont[BoldFont=Arial]{KaiTi_GB2312}
%\setmonofont{NSimSun}




%\XeTeXlinebreaklocale "zh"
%\XeTeXlinebreakskip = 0pt plus 1pt

% Figure, Diagram, Caption settings
%\usepackage{tikz}
%\usetikzlibrary{mindmap,trees}
\usepackage{graphicx}
%\usepackage{graphics}
%\usepackage[hang,small,bf]{caption}
%\setlength{\captionmargin}{50pt}

% Redefine some fonts.
%\newfontfamily\heiti{"黑体"}
%\newfontfamily\fs{"仿宋"}
%\newfontfamily\yahei{"微软雅黑"}


%\usepackage{tikz}
%\usetikzlibrary{mindmap,trees}


%includeonly{}


\graphicspath{{Figures/}{Figures/CPL/}{Figures/DE/}{Figures/P/}}


\usepackage{amsmath}
\begin{document}
\title{Power Spectrum \\ (\textbf{In Progress})}
\author{{\bf MA} Lei  \\
@ Interplanetary Immigration Agency \\
{\small\em \copyright \ Draft date \today}}
\date{}
%\begin{document}
\maketitle

% Redefine some math commands and environments.

\newcommand{\dd}{\mathrm d}
%\newcommand{\HH}{\mathcal H}
%\newcommand{\CN}{{\it Cosmologia Notebook}}
\newenvironment{eqnset}
{\begin{equation}\left \bracevert \begin{array}{l}}
{\end{array} \right. \end{equation}}

\newenvironment{eqn}
{\begin{equation}\left \bracevert \begin{array}{l}}
{\end{array} \right. \end{equation}}


%%%%%%%%%%%%%%%%%%%%%%%%%%%%%%%%%%%%%%%%%%%%%%%%%%%%%%%%
%%%%%%%%%%%%%%    Let's Start Typing     %%%%%%%%%%%%%%%
%%%%%%%%%%%%%%%%%%%%%%%%%%%%%%%%%%%%%%%%%%%%%%%%%%%%%%%%




%%%%%%%%%%    Why power spectra     %%%%%%%%%%


\section{Why power spectra}

In astronomical observations, we observe the properties of objects and their distribution. After we gain these data, we shoud process the data carerfully.

Distributions are always processed in view of power spectrum because power spectrum gives us the two-point function in Fourier space and this makes it possible to evaluate the distribution in terms of modes. For example, power spectrum is often defined as
\begin{equation}
\langle \tilde \delta (\vec k) \tilde \delta (\vec k') \rangle = (2\pi)^3 P(k) \delta^3(\vec k - \vec k')
\end{equation}
in which $\tilde\delta$ is the Fourier transform of $\delta(\vec k) = (n(\vec k) - \bar n)/\bar n$, i.e., density perturbation. $\delta^3(\vec k - \vec k')$ is Dirac function. In this defination, we can see the distribution of different modes in Fourier space. Modes are easily understood because they are plane waves here.











%%%%%%%%%%    What is Power Spectra     %%%%%%%%%%
\section{What is Power Spectra}

In mathematics, power is the square of the coefficients in a Fourier transform with some constants multiplied to them. This is more explicit in a discrete transform. 

The matter perturbation power spectrum reads
\begin{equation}
\langle \tilde \delta (\vec k) \tilde \delta (\vec k') \rangle = (2\pi)^3 P(\vec k) \delta^3(\vec k - \vec k')
\end{equation}
in which $\tilde\delta$ is the Fourier transform of $\delta(\vec k) = (\rho(\vec k) - \bar \rho)/\bar \rho$. and $\rho$ is the relative density of matter. Here at the left is the virance of $\tilde \delta(\vec k)$ which generally means
\begin{equation}
\lim _ {T\rightarrow\infty} \frac 1{2T} \int_{-T}^T\tilde \delta(t) \tilde \delta(t+\tau) dt.
\end{equation}

In the regime of matter power spectrum, we often use the form
\begin{equation}
P(k)=\frac{2\pi^2}{k^3}\delta(k)^2
\end{equation}

The $2\pi^2$ comes from an integral over all the directions of $\vec k$, that is $P(k) = 4\pi  P(\vec k)$.









%%%%%%%%  How to Calculate Power Spectra   %%%%%%%%%%
\section{How to Calculate Power Spectra}


When it comes to the calculation of the power spectrum, we have to find the equation for density evolution in Fourier space. To achieve this, we have to apply Einstein's field equation to the whole universe and the Fourier transform.


\subsection{Background}


Metric for flat FRW universe
\begin{equation}
\bf{\bar g} = 
\begin{pmatrix}
-1 & 0 & 0 & 0 \\
0 & a^2 & 0 & 0 \\
0 & 0 & a^2 & 0 \\
0&0&0& a^2
\end{pmatrix}
\end{equation}

Energy momentum tensor is also needed,
\begin{equation}
	T_{\mu\nu} = (\rho + p)u_\mu u_\nu + p g_{\mu\nu}
\end{equation}





In a FRW universe, the Friedmann equation and acceleration equation which describes the evolution of the background universe is
\begin{eqnarray}
	3(\dot a^2 + k)/ a^2 &=& 8\pi G \sum_i \rho_i \\
	2\ddot a/a + (\dot a^2 + k)/a^2 &=&-8\pi G \sum_i p_i
\end{eqnarray}

In our harmonic model, $\sum_i \rho_i \approx \rho_r + \rho_m + \rho_d$, the lower index $r$ means radiation, $m$ means matter and $d$ stands for dark energy. Moreover, we often define a critical density $\rho_c = \frac{3H_0^2}{8\pi G}$ which is the density that leads to $H=H_0$. Using this critical density, we can define the density fraction $\Omega_i = \frac{\rho_i}{\rho_c}$. Also the Hubble equation $H(a)$ is $H(a) = \dot a/a$.

Since we have strong evidence that $k=0$, the early universe can be roughly described by
\begin{eqnarray}
	3 H(a)^2 &=& 8\pi G (\rho_r + \rho_m) \\
	2 \ddot a/a + H(a)^2 &=& -8\pi G (p_r + p_m)
\end{eqnarray}

When the universe comes into the matter dominated era, the equations can be reduced to
\begin{eqnarray}
	3 H(a)^2 &=& 8\pi G (\rho_m + \rho_d) \\
	2\ddot a/a + H(a)^2 &=& -8\pi G (p_d + p_m)
\end{eqnarray}

To solve the equaiton, we have to find out the pressure of different species.

For baryonic matter, the pressure is essentially zero while for photonic gas, the pressure is $p_r = -4 \rho_r$.
It is more complicated to determine dark energy's equation of state, so we have to put it aside or just denote it with $w$.

Now what are some possible forms of $w$? Two families are often used, i.e.,
\begin{itemize}
	\item [1]
	Redshift is $z=\frac{1- a}{a}$.
\begin{equation}
w(z) = w_0 + w_1 (\frac{z}{1+z})^n
\end{equation}
	\item [2]
\begin{equation}
w(z) = w_0 + w_1 \frac{z}{(1+z)^n}
\end{equation}
\end{itemize}







\subsection{Perturbation}

Cosmological perturbation theory is quite straight forward at first glance. Just find the perturbation of metric and energy momentum tensor and put them together into Einstein's field equation. But, there is a gauge problem in this perturbation formalism. And before we come to the problem, gauge transformation has to be explained.

It is easier to start from ignoring the gauge problem and use the so called Conformal Newtonian gauge to calculate the perturbation.

In this case, the metric is 
\begin{equation}
\bf{g} =
\begin{pmatrix}
	-1-2\Psi(\vec x, t) & 0 & 0 & 0\\
	0 & a^2(1 +  2\Phi(\vec x,t)) & 0 & 0\\
	0 & 0 & a^2(1 +  2\Phi(\vec x,t)) & 0\\
	0 & 0 & 0 & a^2(1 +  2\Phi(\vec x,t))\\
\end{pmatrix}\label{eq-per-metric}
\end{equation}

$\Psi(\vec x,t)$ is the Newtonian potential in a weak field. If you are familar with Schwarzschild spacetime, you can see this imidiately. If we start form the weak field approximation, field equation is $\partial_\lambda \partial^\lambda \phi^{\mu\nu} = 0$. For a static field, this is $\nabla^2 \phi^{\mu\nu} = 0$. The radial sulotion of this equaiton, is Newtonian potential, that is 00 component of the metric is $-2GM/r$. $\Phi(\vec x,t)$ is the perturbation to space.

Here is the problem. In \ref{eq-per-metric}, $\Phi$ and $\Psi$ relies on the the choice of a coordinate system, which means we choose some kind of threading and slicing of spacetime. When we choose another coordinate system, the equations may change under this coordinate transformation. Now we see this is a bad formalism, even in harmonic analysis. Harmonic analysis is our first mathematics excerse before we start to deal with the gauge invariant perturbation. Let's see it.

\subsubsection{Harmonic Analysis}

For vector $V_i$ and rank 2 tensor $H_{ij}$, we can always write them into certain parts.
\begin{eqnarray}
	V_i &=& \nabla_i \phi + B_i\\
	H_{ij} &=& H_L \gamma_{ij} + \left(\nabla_i\nabla_j - \frac{1}{3}\Delta\gamma_{ij}\right) H_{T} + \frac{1}{2}\left(H^{(V)}_{i|j} + H^{(V)}_{j|i} \right) + H^{(T)}_{ij}
\end{eqnarray}
Some properties of the quantities above are
\begin{eqnarray}
	B^i_{|i}=0, &&\qquad\text{transverse} \\
	H^{(V)|i}_i = 0, &&\qquad\text{transverse} \\
	H^{(T)i}_i = 0,&&\qquad \text{traceless} \\
	H^{(T)j}_{i|j} = 0, &&\qquad \text{transverse} \\
	H_L, H_T  &&\qquad \text{spin-0 part} \\
	H^{(V)}_i  	&&\qquad \text{spin-1 part} \\
	H^{(T)}_{ij}, &&\qquad \text{spin-2 part}
\end{eqnarray}


If scalar modes are the only concern, we can drop vector and tensor modes because scalar, vector and tensor modes evolve independently.


For scalar variables, harmonic analysis is done on basis $e^{i\vec k\cdot \vec x}$. But it requires a spherical harmonic $Y$ if we need to expand scalars in on the surface fo a sphere. Since spherical harmonic $Y$ can be reduced to Fourier basis at some limits, we would perform the harmonic analysis using $Y$, with $k^2=l(l+2)$.


We can expand all scalar mode variables here on spherical harmonic Y, eigenfunctions of Laplacian $(\Delta + k^2) Y =0$, and its gradients $Y_{j}\equiv -k^{-1} Y_{|j}$ and $Y_{ij}\equiv k^{-2}Y_{|ij} +\frac13\gamma_{ij}Y$, if we do not deal with vector and tensor modes. That is
\begin{eqnarray}
	V_i &=& V Y_i \\
	H_{ij} &=& H_L \gamma_{ij} Y + H_T Y_{ij}
\end{eqnarray}





The metric and its perturbation becomes
\begin{eqnarray}
\tilde g_{00} &=&	-a^2( 1 + 2 A Y) \\
\tilde g_{0j} &=& -a^2 B Y_j \\
\tilde g_{ij} &=& a^2(\gamma_{ij} + 2 H_L Y \gamma_{ij} + 2 H_T Y_{ij})
\end{eqnarray}

Inverse of it
\begin{eqnarray}
\tilde g^{00}&=& -a^{-2}(1-2A Y) \\
\tilde g^{0j}&=& -a^{-2} B Y^i \\
\tilde g^{ij}&=& a^{-2}(\gamma^{ij}-2H_L Y \gamma^{ij}-2H_T Y^{ij} )
\end{eqnarray}


We also have energy momentum tensor to expand. Suppose we have perfect fluid as the source of gravity. First rule, we always talk about rest observers. Since energy momentum tensor is related to observers, we have to evaluate this tensor by velocity.
\begin{eqnarray}
	\tilde T^\mu_\nu \tilde u^\nu &=& -\tilde \rho \tilde u^\mu \\
	\tilde u_\mu \tilde u^\mu &=& -1
\end{eqnarray}

A careful calculation shows 
\begin{eqnarray}
T^0_{\phantom 0 0} &=& -\rho (1+\delta Y)  \\\
T^0_{\phantom 0j } &=& (\rho + p )(v-B)Y_j \\
T^j_{\phantom j0 }&=& -(\rho + p) v Y^j \\
T^{i}_{\phantom i j} &=& p(\delta^i_j + \pi_L \delta^i_j Y +\pi_T Y^i_j)
\end{eqnarray}








\subsubsection{Gauge Transformation}

The second math excerse is gauge transformation.

Gauge transformation here is infinitesimal coordination transformation from the view of coordinate transformation, or more precisely, the Lie derivitive along some vector $X=T\partial_t+L^i\partial_i$. To be simple, we start from infinitesimal coordinate transformation,
\begin{eqnarray}
	\bar\eta &=& \eta + T(\eta) Y \\
	\bar x^j &=& x^j + L(\eta) Y^j
\end{eqnarray}

Then metric transforms like
\begin{eqnarray}
	\bar{\tilde g} _{\mu\nu}(\eta,x^j) &=& \frac{\partial x^\alpha}{\partial{\bar x^\mu}} \frac{\partial x^\beta }{\partial x^\nu} \tilde g_{\alpha\beta}(\eta-T Y, x^j - L Y^j)  \\
	&=& \tilde g_{\mu\nu}(\eta,x^j) + g_{\alpha\nu}\delta x^\alpha_{\phantom\alpha,\mu} + g_{\alpha\mu}\delta x^{\alpha}_{\phantom \alpha,\nu} - g_{\mu\nu,\lambda}\delta x^\lambda
\end{eqnarray}


Put everything (perturbated metric and the gauge transformation) in, we can find out that
\begin{eqnarray}
	\bar A&=& A- T' - (a'/a)T \\
	\bar B&=& B+L' + k T \\
	\bar H_L &=& H_L - (k/n) L - (a'/a)T \\
	\bar H_T &=& H_T + k L 
\end{eqnarray}

Transformation for energy momentum tensor
\begin{eqnarray}
	\bar v &=& v +L' \\
	\bar{\tilde \rho}(\eta) &=& \tilde \rho(\eta) - \rho' T Y \\
	\bar \delta &=& \delta + n(1+w)(a'/a) T   \\
	\bar \pi_L &=& \pi_L + \frac{c_s^2}{w} n(1+w)\frac{a'}{a} T \\
	\bar \pi_T &=& \pi_T
\end{eqnarray}

The sound speed is $c_s^2\equiv \dot p/\dot \rho$.

Our formalism is not invariant under such gauge transformations. Thus our task is to find out gauge invariant variables and use them to do the theoretical calculation.



\subsubsection{Gauge Invariant Variables}

We can construct many gauge variables use the gauge transformation properties. But some of them are more convinient for certain calculations.

For example we can choose Bardeen potentials

\begin{eqnarray}
	\Phi &=& H_L + \frac13 H_T + \frac1k \mathcal H (B - \frac{1}{k}H_T' ) \\
	\Psi &=& A + \frac 1 k \mathcal H (B - \frac{1}{k}H_T'  ) + \frac{1}{k}  (B' - \frac1k H_T '')
\end{eqnarray}

We can also define variable using energy momentum tensor,
\begin{eqnarray}
	\Gamma &\equiv & \pi_L - \frac{c_s^2}{w} \delta \\
	\Pi & \equiv & \pi_T
\end{eqnarray}

But we still do not have variable corresponding to $\delta$ and $v$. This requires some mixture of metric. (This is the wretched spacetime in General Relativity. \footnote{\color{red}{Need more information for this point.}} )

Velocity is
\begin{eqnarray}
	V &\equiv & v - k^{-1} H_T' \\
\end{eqnarray}

The variable for density is not completely fixed,
\begin{eqnarray}
	\Delta_s&\equiv & \delta + 3 (1+w)(a'/a)k^{-1} \sigma_g \\
	\Delta_g &\equiv & \Delta_s + 3 (1+w) \Phi = \delta + 3(1+w)\mathcal R \\
	\Delta &\equiv & \Delta_s  + 3(1+w)(a'/a) k^{-1} V \\
	&=& \Delta_g - 3(1+w)\Phi + 3(1+w)(a'/a)k^{-1}V \\
	&=& \delta 3(1+w) (a'/a)k^{-1}(v-B)
\end{eqnarray}

Perturbation of shear is $\sigma_g\equiv k^{-1} H_T' - B$. The perturbation of intrinsic curvature on a constant time hypersurface is $\mathcal R = H_L + \frac13 H_T$ which can be seen by calculating the perturbation of scalar curvature $R$. As for the meaning of other variables, we can see them in different gauges.



\subsubsection{Perturbated Equaitons}

Perturbated $G_{0\mu} = 8\pi G T_{0\mu}$ part of Einstein equations is \footnote{Kodama and Sasaki gives a method to use two auxilliary gauge invariant variables $\mathcal A$ and $\mathcal B$ to calculate these equations. They are very convinient.}
\begin{eqnarray}
	4\pi G a^2 \rho \Delta &=& - k^2 \Phi, \qquad\text{Poisson equation}\\
	4\pi G a^2 (\rho + p ) V &=& k(\mathcal H \Psi + \dot\Phi)
\end{eqnarray}

$G_{ij} = 8\pi G T_{ij}$ becomes
\begin{eqnarray}
	k^2(\Phi - \Psi) &=& 8\pi G a^2 p \Pi \\
	\ddot \Phi + 2\mathcal H \dot \Phi + \mathcal H \dot \Psi + (2\dot {\mathcal H} + \mathcal H^2 - \frac{k^2}{3} )\Psi &=& 4\pi G a^2 \rho (\frac13 \Delta + c_s^2 \Delta_s + w \Gamma)
\end{eqnarray}


\subsubsection{Different Gauges}

The meaning of variables can be made clear with a graph on Page 336 of \emph{Cosmological Inflation and Large-Scale Structure}.

We can see that $v$ is the angle between the threading and the velocity of fluid and $B$ is the angle between threading and normal of slicing.

There is too much freedom ($T$ and $L$ are introduced in the gauge transformation) in this formalism, we have to fix it (eliminate $T$ and $L$) by choosing some particular gauge.

Fixing a gauge means choosing some particular slicing or threading, such as proper-time slicing ($A=0$) and normal threading ($B=0$) which gives us the synchronous gauge, zero time derivative of amplitude of anisotropic distortion of each constant time hypersurface ($H_T' = 0$) and normal threading ($B=0$) which gives the logitudinal gauge.


Once the gauge is fixed, we can write down the perturbation equaitions and try to solve them.







%%%%%%%%  Harrison-Zeldovich Prescription   %%%%%%%%%%
\section{Harrison-Zeldovich Prescription}






%%%%%%%%  Evolution of Power Spectra With HZ Prescription   %%%%%%%%%%
\section{Evolution of Power Spectra With HZ Prescription}





%%%%%%%%  Models and Parameters   %%%%%%%%%%
\section{Models and Parameters}




%%%%%%%%  Transitions   %%%%%%%%%%
\section{Transitions}


















\end{document}